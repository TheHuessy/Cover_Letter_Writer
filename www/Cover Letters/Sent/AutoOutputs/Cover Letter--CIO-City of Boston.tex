\documentclass[]{article}
\usepackage{lmodern}
\usepackage{amssymb,amsmath}
\usepackage{ifxetex,ifluatex}
\usepackage{fixltx2e} % provides \textsubscript
\ifnum 0\ifxetex 1\fi\ifluatex 1\fi=0 % if pdftex
  \usepackage[T1]{fontenc}
  \usepackage[utf8]{inputenc}
\else % if luatex or xelatex
  \ifxetex
    \usepackage{mathspec}
  \else
    \usepackage{fontspec}
  \fi
  \defaultfontfeatures{Ligatures=TeX,Scale=MatchLowercase}
\fi
% use upquote if available, for straight quotes in verbatim environments
\IfFileExists{upquote.sty}{\usepackage{upquote}}{}
% use microtype if available
\IfFileExists{microtype.sty}{%
\usepackage{microtype}
\UseMicrotypeSet[protrusion]{basicmath} % disable protrusion for tt fonts
}{}
\usepackage[margin=1in]{geometry}
\usepackage{hyperref}
\hypersetup{unicode=true,
            pdfborder={0 0 0},
            breaklinks=true}
\urlstyle{same}  % don't use monospace font for urls
\usepackage{graphicx,grffile}
\makeatletter
\def\maxwidth{\ifdim\Gin@nat@width>\linewidth\linewidth\else\Gin@nat@width\fi}
\def\maxheight{\ifdim\Gin@nat@height>\textheight\textheight\else\Gin@nat@height\fi}
\makeatother
% Scale images if necessary, so that they will not overflow the page
% margins by default, and it is still possible to overwrite the defaults
% using explicit options in \includegraphics[width, height, ...]{}
\setkeys{Gin}{width=\maxwidth,height=\maxheight,keepaspectratio}
\IfFileExists{parskip.sty}{%
\usepackage{parskip}
}{% else
\setlength{\parindent}{0pt}
\setlength{\parskip}{6pt plus 2pt minus 1pt}
}
\setlength{\emergencystretch}{3em}  % prevent overfull lines
\providecommand{\tightlist}{%
  \setlength{\itemsep}{0pt}\setlength{\parskip}{0pt}}
\setcounter{secnumdepth}{0}
% Redefines (sub)paragraphs to behave more like sections
\ifx\paragraph\undefined\else
\let\oldparagraph\paragraph
\renewcommand{\paragraph}[1]{\oldparagraph{#1}\mbox{}}
\fi
\ifx\subparagraph\undefined\else
\let\oldsubparagraph\subparagraph
\renewcommand{\subparagraph}[1]{\oldsubparagraph{#1}\mbox{}}
\fi

%%% Use protect on footnotes to avoid problems with footnotes in titles
\let\rmarkdownfootnote\footnote%
\def\footnote{\protect\rmarkdownfootnote}

%%% Change title format to be more compact
\usepackage{titling}

% Create subtitle command for use in maketitle
\newcommand{\subtitle}[1]{
  \posttitle{
    \begin{center}\large#1\end{center}
    }
}

\setlength{\droptitle}{-2em}

  \title{}
    \pretitle{\vspace{\droptitle}}
  \posttitle{}
    \author{}
    \preauthor{}\postauthor{}
    \date{}
    \predate{}\postdate{}
  
\usepackage{fancyhdr}
\pagestyle{fancy}
\fancyfoot[CO,CE]{This cover letter was generated using a Shiny App I wrote that creates cover letters using RMarkdown. Think of what I could create if I worked for you!}

\begin{document}

City of Boston

1 City Hall Plaza

Boston, MA 02115

\hypertarget{section}{%
\subparagraph{}\label{section}}

December 04, 2018

Dear Hiring Committee,

I am seeking employment at City of Boston because I am looking to
advance my career in data analysis and lend my skills and talents to a
focused investment company. I would like to hone my skills in research,
data analysis, programming, and statistical writing while exploring
other applications of big data outside of the field of public policy and
gain experience in the financial sector. Working at City of Boston would
give me all of these opportunities and enable me to continue my career
in data science.

I am applying for the CIO position because I feel that my background and
skillsets would make me an asset to the team. At the School of Public
Policy and Urban Affairs, I worked with many principal investigators on
a myriad of grand funded projects covering almost all aspects of public
policy. I was hired initially as an undergrad, for a summer job, and my
quantitative and technical talents were quickly recognized and I found
myself being pulled onto more and more projects to provide any and all
data manipulation and analysis. AND I'M THE COOLEST!

I am very well versed in R, which began as an intro class when I started
my Masters and is now an integral part of my everyday work. Where some
wanted to keep using excel or SPSS to run analysis, I saw the
opportunity to use R and develop not only quality statistical analysis,
but also better visual representations of the data from simple ggplot
outputs to shiny apps written to showcase geographical differences in
data at different levels. Due in part to my policy background and skills
in R, I was hired by the Boston Area Research Initiative (BARI) to head
up housing and transportation data projects. Since January 2018, I have
built an automated web scraper that collects, stores, and cleans
Craigslist data, and most importantly I built, set up, and maintain a
local storage server for BARI which is now used to back up all big data
files and host local geocoding software.

Now, as I near the end of my graduate studies, I feel that I have
outgrown my former roles and want to seek a new challenge, working with
even larger datasets, developing insights that can further the mission
of the company. I want to brave the new frontier in my budding data
science career and working at City of Boston would allow me to do so. I
take genuine joy in researching, cleaning data, analyzing the datasets,
and reporting the story within the data.

I look forward to hearing from you at your convenience. Thank you for
your consideration.

Sincerely,

James Huessy


\end{document}
